% Options for packages loaded elsewhere
\PassOptionsToPackage{unicode}{hyperref}
\PassOptionsToPackage{hyphens}{url}
%
\documentclass[10pt,a4paper]{article}
\usepackage[left=25mm,right=25mm]{geometry}
\usepackage{amsmath}
\usepackage{amsfonts}
\usepackage{amssymb}

\author{}
\date{}



\usepackage{listings}
\usepackage{color}

\definecolor{dkgreen}{rgb}{0,0.6,0}
\definecolor{gray}{rgb}{0.5,0.5,0.5}
\definecolor{mauve}{rgb}{0.58,0,0.82}

\lstset{frame=tb,
  language=C,
  aboveskip=3mm,
  belowskip=3mm,
  showstringspaces=false,
  columns=flexible,
  basicstyle={\small\ttfamily},
  numbers=none,
  numberstyle=\tiny\color{gray},
  keywordstyle=\color{blue},
  commentstyle=\color{dkgreen},
  stringstyle=\color{mauve},
  breaklines=true,
  breakatwhitespace=true,
  tabsize=3
}

\usepackage{multicol}
\usepackage{graphicx}
\usepackage{epstopdf}

\epstopdfDeclareGraphicsRule{.gif}{png}{.png}{convert gif:#1 png:\OutputFile}
\AppendGraphicsExtensions{.gif}
\usepackage{chngcntr}
\counterwithin*{equation}{section}
\counterwithin*{equation}{subsection}
\usepackage{amsmath}

\usepackage{float} 
\usepackage{hyperref}
\usepackage{amsmath}
\let\oldsubsection\subsection
\renewcommand{\subsection}{%
    \setcounter{equation}{0}%
    \oldsubsection%
}

\begin{document}


\begin{flushleft}
\begin{LARGE}EE 435 Homework 2 Spring 2024
\end{LARGE}
\\Jonathan Hess
\\\href{https://github.com/Jetsama/EE435/tree/main/HW2}{GitHub Page}
\end{flushleft}


\subsection*{Problem 1 and 2}
Consider the following operational amplifier. The goal is to obtain an
expression for the small-signal output voltage in terms of the input variables INV+ and INV-.

\subsubsection*{a)}
Write a complete set of small-signal equations that can be solved to obtain
\(V_{OUT}\). Assume the small-signal parameter \(g_o\) is present in all MOS devices.

\subsubsection*{b)}
Solve these equations by hand for \(V_{OUT}\). If you do not have a solution at
the end of ½ hour, stop, and comment on your progress and the amount of
effort that you believe would be required to finish the solution.

\subsubsection*{c)}
Obtain a parametric (symbolic) solution for the transfer function \(V_{OUT}/V_{IN}\)
from this set of equations with MATLAB. How many total product terms appear in this
solution? In this part, \(V_{IN} = V_{IN+} - V_{IN-}\).

\subsubsection*{d)}
Simplify the solution obtained with MATLAB under the assumption that all \(g_o\) terms are small compared to \(g_m\) terms.

\subsection*{Problem 3}
A transresistance amplifier with a gain \(R_T\) is shown. Derive an expression
for the voltage gain of the amplifier as a function of the transresistance gain \(R_T\) and determine what that reduces to if \(R_T\) is very large.
\includegraphics[width=6in]{images/Problem3.png} \\


In this circuit we are using a transresistance amplifier which has a gain of $V_{out} = R_f * i_{in}$. The $i_{in}$ for this circuit is $\frac{V_{in}}{R_1}$ because $V_{neg}$ is a virtual ground. We can then do a KVL.

\begin{equation}
\label{eq:trans}

\end{equation}

\subsection*{Problem 4}
Assume the amplifier shown below is designed in a 0.18$\mu$ CMOS process.
Assume also that \(V_{DD} = 1V\), \(V_{SS} = -1V\), and \(I_{DQ} = 4mA\).

\subsubsection*{a)}
Analytically determine the \(W\) and \(L\) needed to establish a quiescent output
voltage of 0.5V when \(V_{INQ} = 0.5V\).

\subsubsection*{b)}
Verify the transfer characteristics by Spice simulation.

\subsubsection*{c)}
Analytically determine the dc voltage gain at the Q-point established in a).

\subsubsection*{d)}
Using SPICE, obtain a plot of the small signal voltage gain versus the
quiescent output voltage.

\subsection*{Problem 5 and 6}
Design a 5T op amp to have a dc gain of 50dB and a GB of 2MHz
in the ON 0.5$\mu$m CMOS process. Assume \(V_{DD} = 3.5V\) and \(C_L = 1pF\).
Assume also that the bias voltages \(V_{B1}\) and \(V_{B2}\) can be precisely set so that a CMFB circuit is not needed.
Verify the gain and the GB of your design with a SPICE simulation.

\subsection*{Problem 7}
Determine the common-mode input range and the output signal swing of
the amplifier you designed in the previous problem.

\subsection*{Problem 8}
Assume that the op amp is a single-pole amplifier with gain given by the
expression \(A(s) = \frac{A_{GB}}{1 + \frac{s}{\omega_A}}\), where the gain-bandwidth product of the op amp is
\(\omega_A A_{GB} = \omega\). Assuming that the frequency-dependent gain of the op amp can be modeled as
\(GBA(s) = \frac{A_{GB}}{s}\), determine the transfer function
\(\frac{V_{OUT}}{V_{IN}}\) for the following two amplifiers.

\subsubsection*{a)}
\[
\frac{V_{OUT}}{V_{IN}} = \frac{A_{GB}}{s} \times \frac{1}{1 + \frac{1}{2}s}
\]

\subsubsection*{b)}
\[
\frac{V_{OUT}}{V_{IN}} = \frac{A_{GB}}{s} \times \frac{1}{1 + \frac{1}{10}s}
\]

\subsubsection*{c)}
With the same op amp model used in part a), analytically determine the 3dB
bandwidth of the following two amplifiers.

\subsubsection*{d)}
Derive the gain of the inverting feedback amplifier in terms of \(A_{V}\) and \(\beta\) and
comment on why it does not look like the standard feedback equation.

\subsection*{Problem 9}
It was stated in class that all even-ordered distortion terms introduced by
the amplifier vanish in symmetric fully differential amplifiers. Prove this fact.

\subsection*{Problem 10 (Extra Credit)}
The “dead network” of a circuit is obtained by setting all small-signal inputs to 0. That is, by replacing all ac voltage sources with short circuits and all ac current sources with open circuits. The $\beta$ of a feedback amplifier is a characteristic of the “dead network”. Consider the basic inverting and noninverting feedback amplifiers shown below. These are widely used as small-signal voltage amplifiers.

\subsubsection*{a)}
Show that they both have the same “dead network”.

\subsubsection*{b)}
The \(\beta\) of the two amplifiers shown is \(\frac{1}{1 + \frac{R_1}{R_2}}\). Show that the gain of the
noninverting feedback amplifier can be expressed by the standard feedback
equation
\[
A = \frac{V_{OUT}}{V_{IN}} = \frac{1 + \frac{R_2}{R_1}}{\frac{R_2}{R_1}}
\]

\subsubsection*{c)}
Take the limit as \(A_V\) goes to $\infty$ for the gain derived in part b) and compare with
that derived in EE 230 for the gain of the noninverting feedback amplifier.

\subsubsection*{d)}
Derive the gain of the inverting feedback amplifier in terms of \(A_V\) and \(\beta\) and
comment on why it does not look like the standard feedback equation.

\end{document}
