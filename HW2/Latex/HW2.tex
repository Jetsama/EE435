% Options for packages loaded elsewhere
\PassOptionsToPackage{unicode}{hyperref}
\PassOptionsToPackage{hyphens}{url}
%
\documentclass[10pt,a4paper]{article}
\usepackage[left=25mm,right=25mm]{geometry}
\usepackage{amsmath}
\usepackage{amsfonts}
\usepackage{amssymb}

\author{}
\date{}



\usepackage{listings}
\usepackage{color}

\definecolor{dkgreen}{rgb}{0,0.6,0}
\definecolor{gray}{rgb}{0.5,0.5,0.5}
\definecolor{mauve}{rgb}{0.58,0,0.82}

\lstset{frame=tb,
  language=C,
  aboveskip=3mm,
  belowskip=3mm,
  showstringspaces=false,
  columns=flexible,
  basicstyle={\small\ttfamily},
  numbers=none,
  numberstyle=\tiny\color{gray},
  keywordstyle=\color{blue},
  commentstyle=\color{dkgreen},
  stringstyle=\color{mauve},
  breaklines=true,
  breakatwhitespace=true,
  tabsize=3
}

\usepackage{multicol}
\usepackage{graphicx}
\usepackage{epstopdf}

\epstopdfDeclareGraphicsRule{.gif}{png}{.png}{convert gif:#1 png:\OutputFile}
\AppendGraphicsExtensions{.gif}
\usepackage{chngcntr}
\counterwithin*{equation}{section}
\counterwithin*{equation}{subsection}
\usepackage{amsmath}

\usepackage{float} 
\usepackage{hyperref}
\usepackage{amsmath}
\let\oldsubsection\subsection
\renewcommand{\subsection}{%
    \setcounter{equation}{0}%
    \oldsubsection%
}
\begin{document}


\begin{flushleft}
\begin{LARGE}EE 435 Homework 2 Spring 2024
\end{LARGE}
\\Jonathan Hess
\\\href{https://github.com/Jetsama/EE435/tree/main/HW2}{GitHub Page}
\end{flushleft}


\subsection*{Problem 1 and 2}
Consider the following operational amplifier. The goal is to obtain an
expression for the small-signal output voltage in terms of the input variables INV+ and INV-.\\
\includegraphics[width=3in]{images/Problem1.png} \\


\subsubsection*{a)}
Write a complete set of small-signal equations that can be solved to obtain
\(V_{OUT}\). Assume the small-signal parameter \(g_o\) is present in all MOS devices.\\


\includegraphics[width=3in]{images/p1p1.png} \\
\includegraphics[width=3in]{images/p1p2.png} \\

Three equations are created via KCL analysis of the small signal model. They are as follows:\\

\[(V_{\text{out}} - V_{a}) \cdot g_0 + V_{\text{out}} \cdot g_0 + (V_{\text{inN}} - V_{a}) \cdot g_m + V_m \cdot g_m = 0\]

\[(V_m - V_{a}) \cdot g_0 + V_m \cdot g_0 + (V_{\text{inP}} - V_{a}) \cdot g_m + V_m \cdot g_m = 0\]

\[V_a \cdot g_0 + (V_a - V_m) \cdot g_0 + (V_a - V_{\text{out}}) \cdot g_0 - (V_{\text{inP}} - V_{a}) \cdot g_m - (V_{\text{inN}} - V_{a}) \cdot g_m = 0\]

\subsubsection*{b)}
Solve these equations by hand for \(V_{OUT}\). If you do not have a solution at
the end of ½ hour, stop, and comment on your progress and the amount of
effort that you believe would be required to finish the solution.\\

\textbf{INCORRECT}\\
NOTE: this is attempt was incorrect because I used go as resistance rather than conductance

\includegraphics[width=3in]{images/p1p3.png} \\
\begin{equation}
\frac{V_{out}}{g_0} + (V_m) gm = \frac{V_{out} - V_a }{g_0} + V_{in-}g_m 
\end{equation}

\begin{equation}
\frac{V_{out}}{g_0} \frac{V_{out} - V_a }{g_0} = (V_m) gm + V_{in-}g_m 
\end{equation}
\begin{equation}
V_{out} * \frac{2}{g_0} = (V_{in-} + V_m - V_a)g_m +  \frac{V_{a}}{g_0}
\end{equation}

\begin{equation}
V_{out} = \frac{(V_{in-} + V_m - V_a)g_m}{2} +  \frac{V_{a}}{2}
\end{equation}



\begin{equation}
V_{m} = (\frac{-V_{out}}{g_0} + V_{in-}g_m) * \frac{1}{\frac{1}{g_0} - g_m + 2g_m^2g_0}
\end{equation}

\begin{equation}
V_{a} = V_m + V_{out} + 2(g_m)(V_m)(g_0)
\end{equation}

\begin{equation}
V_{a} = V_{out} + V_m( 1+ 2(g_m)(g_0))
\end{equation}


Now replace all $V_a$ in the VOUT equation.

\begin{equation}
V_{out} = \frac{(V_{in-} + V_{out} - 2(g_m)(V_m)(g_0))g_m}{2} +  \frac{V_m + V_{out} + 2(g_m)(V_m)(g_0)}{2}
\end{equation}
Now replace all $V_m$ in the  equation.
\begin{multline}
V_{out} = \frac{(V_{in-} + V_{out} - 2(g_m)((\frac{-V_{out}}{g_0} + V_{in-}g_m) * \frac{1}{\frac{1}{g_0} - g_m + 2g_m^2g_0})(g_0))g_m}{2}\\ +  \frac{(\frac{-V_{out}}{g_0} + V_{in-}g_m) * \frac{1}{\frac{1}{g_0} - g_m + 2g_m^2g_0} + V_{out} + 2(g_m)((\frac{-V_{out}}{g_0} + V_{in-}g_m) * \frac{1}{\frac{1}{g_0} - g_m + 2g_m^2g_0})(g_0)}{2}
\end{multline}
At this point I spent way to much time on this problem. So I moved onto part c. I then realized that I was using g0 as a resistance rather than conductance.\\
\textbf{CORRECTED}\\



\subsubsection*{c)}
Obtain a parametric (symbolic) solution for the transfer function \(V_{OUT}/V_{IN}\)
from this set of equations with MATLAB. How many total product terms appear in this
solution? In this part, \(V_{IN} = V_{IN+} - V_{IN-}\).\\

\begin{lstlisting}
syms Vm Va Vout go gm Vin VinN VinP

eq1 = (Vout-Va)*go + Vout*go + (VinN-Va)*gm +Vm*gm  == 0;
eq2 = (Vm-Va)*go + Vm*go + (VinP-Va)*gm +Vm*gm  == 0;
eq3 = Va*go + (Va-Vm)*go +(Va-Vout)*go - (VinP-Va)*gm - (VinN-Va)*gm == 0;
eq4 = Vin == VinP-VinN;

out = solve([eq1,eq2,eq3,eq4], [Vout,Vm,VinP,VinN,Va])
out.Vout

\end{lstlisting}
\begin{lstlisting}

ans =
 
(2*Vin*gm^2 + Vin*go*gm)/(4*go*(gm + go))\end{lstlisting}


\begin{equation}
\frac{V_{out}}{V_{in}} = \frac{(2 * g_m^2) + (g_m g_0) }{4 g_0 (g_m + g_0)}
\end{equation}

\subsubsection*{d)}
Simplify the solution obtained with MATLAB under the assumption that all \(g_o\) terms are small compared to \(g_m\) terms.\\

\begin{equation}
\frac{V_{out}}{V_{in}} =\frac{g_m}{2 g_0}
\end{equation}











\subsection*{Problem 3}
A transresistance amplifier with a gain \(R_T\) is shown. Derive an expression
for the voltage gain of the amplifier as a function of the transresistance gain \(R_T\) and determine what that reduces to if \(R_T\) is very large.\\
\includegraphics[width=6in]{images/Problem3.png} \\


In this circuit we are using a transresistance amplifier which has a gain of $V_{out} = R_f * i_{in}$. The $i_{in}$ for this circuit is $\frac{V_{in}}{R_1}$ because $V_{neg}$ is a virtual ground. We can then do a KVL. 
\begin{equation}
\frac{V_{in}}{R_1} + \frac{V_{out}}{R_2} - i_{in}= 0
\end{equation}

\begin{equation}
\frac{V_{in}}{R_1} + \frac{V_{out}}{R_2} - \frac{V_{out}}(R_f)= 0
\end{equation}

\begin{equation}
\frac{V_{in}}{R_1} =  V_{out}  * \frac{R_2 -R_f}{R_2*R_f}
\end{equation}

\begin{equation}
\frac{V_{out}}{V_{in}} =  \frac{R_2*R_f}{R_1(R_2 -R_f)}
\end{equation}



\begin{equation}
lim_{R_f \to \infty} \frac{R_2*R_f}{R_1(R_2 -R_f)} = \frac{R_2}{R_1}
\end{equation}
	
	
	
	
	
	
	
	
	

\subsection*{Problem 4}
Assume the amplifier shown below is designed in a 0.18$\mu$ CMOS process.
Assume also that \(V_{DD} = 1V\), \(V_{SS} = -1V\), and \(I_{DQ} = 4mA\).\\
\includegraphics[width=3in]{images/Problem4.png} \\

\subsubsection*{a)}
Analytically determine the \(W\) and \(L\) needed to establish a quiescent output
voltage of 0.5V when \(V_{INQ} = 0.5V\).\\

Since we are using a mosfet in linear (non saturation region), we can use the following equations:

\begin{equation}
i_{d} = K * (2(V_{GS} - V_{TN}) V_{DS} - V_{DS}^2)
\end{equation}
Where
\begin{equation}
K =\frac{1}{2}(\mu_n C_{ox})\frac{W}{L}
\end{equation}


\begin{equation}
V_{out} = V_{DS} - V_{SS}  = 0.5 V
\end{equation}

\begin{equation}
V_{DS} = 1.5 V
\end{equation}

\begin{equation}
V_{in} = V_{GS} - V_{SS} = 0.5 V
\end{equation}
\begin{equation}
V_{GS} = 1.5 V
\end{equation}

To get the W/L we can use the current:

\begin{equation}
i_{d} = K * (2(V_{GS} - V_{TN}) V_{DS} - V_{DS}^2)	
\end{equation}
To get the VTN, I used the reference sheet for the process. 
\begin{equation}
V_{TN} =  0.5V
\end{equation}


The output current of the amplifier is:
\begin{equation}
i_{d} = 4mA = K * (2(1.5 -0.5)(1.5V) - (1.5V)^2)
\end{equation}

\begin{equation}
i_{d} = 4mA = K * (2(1)(1.5V) - (1.5V)^2)
\end{equation}

\begin{equation}
4mA = K * (3 - 2.25)V
\end{equation}

\begin{equation}
4*10^{-3} A = K * 0.75V
\end{equation}

\begin{equation}
\frac{16}{3}*10^{-3} = K = \frac{1}{2}(\mu_n C_{ox})\frac{W}{L} = 171.8 \frac{uA}{V^2} \frac{W}{L}
\end{equation}

\begin{equation}
\frac{W}{L} = \frac{80000}{2577} \approx 31.04
\end{equation}



\subsubsection*{b)}
Verify the transfer characteristics by Spice simulation.

\subsubsection*{c)}
Analytically determine the dc voltage gain at the Q-point established in a).

\subsubsection*{d)}
Using SPICE, obtain a plot of the small signal voltage gain versus the
quiescent output voltage.

\subsection*{Problem 5 and 6}
Design a 5T op amp to have a dc gain of 50dB and a GB of 2MHz
in the ON 0.5$\mu$m CMOS process. Assume \(V_{DD} = 3.5V\) and \(C_L = 1pF\).
Assume also that the bias voltages \(V_{B1}\) and \(V_{B2}\) can be precisely set so that a CMFB circuit is not needed.
Verify the gain and the GB of your design with a SPICE simulation.\\
\includegraphics[width=4in]{images/Problem5.png} \\

We can use the equations given in the lecture slides.\\
\includegraphics[width=4in]{images/5TOpamp.png} \\
\includegraphics[width=4in]{images/5TOpampPractical.png} \\

First convert the $50dB = 10 log(\frac{V_{out}}{V_{in}})$. We then get that $A = 10^5$.


\begin{equation}
A = \frac{1}{\lambda_1 + \lambda_2} \frac{1}{V_{EB1}} = 10^5 =
\end{equation}


\begin{equation}
g_{mn} = (\mu_n C_{ox})\frac{W}{L} = 57.8*2 \frac{W}{L}
\end{equation}



\subsection*{Problem 7}
Determine the common-mode input range and the output signal swing of
the amplifier you designed in the previous problem.

\subsection*{Problem 8}
Assume that the op amp is a single-pole amplifier with gain given by the
expression \(A(s) = \frac{A_{GB}}{1 + \frac{s}{\omega_A}}\), where the gain-bandwidth product of the op amp is
\(\omega_A A_{GB} = \omega\). Assuming that the frequency-dependent gain of the op amp can be modeled as
\(GBA(s) = \frac{A_{GB}}{s}\), determine the transfer function
\(\frac{V_{OUT}}{V_{IN}}\) for the following two amplifiers.
\includegraphics[width=6in]{images/Problem8.png} \\

\subsubsection*{a)}
\[
\frac{V_{OUT}}{V_{IN}} = \frac{A_{GB}}{s} \times \frac{1}{1 + \frac{1}{2}s}
\]

\subsubsection*{b)}
\[
\frac{V_{OUT}}{V_{IN}} = \frac{A_{GB}}{s} \times \frac{1}{1 + \frac{1}{10}s}
\]

\subsubsection*{c)}
With the same op amp model used in part a), analytically determine the 3dB
bandwidth of the following two amplifiers.

\subsubsection*{d)}
Derive the gain of the inverting feedback amplifier in terms of \(A_{V}\) and \(\beta\) and
comment on why it does not look like the standard feedback equation.

\subsection*{Problem 9}
It was stated in class that all even-ordered distortion terms introduced by
the amplifier vanish in symmetric fully differential amplifiers. Prove this fact.

\subsection*{Problem 10 (Extra Credit)}
The “dead network” of a circuit is obtained by setting all small-signal inputs to 0. That is, by replacing all ac voltage sources with short circuits and all ac current sources with open circuits. The $\beta$ of a feedback amplifier is a characteristic of the “dead network”. Consider the basic inverting and noninverting feedback amplifiers shown below. These are widely used as small-signal voltage amplifiers.\\
\includegraphics[width=6in]{images/Problem10.png} \\

\subsubsection*{a)}
Show that they both have the same “dead network”.

\subsubsection*{b)}
The \(\beta\) of the two amplifiers shown is \(\frac{1}{1 + \frac{R_1}{R_2}}\). Show that the gain of the
noninverting feedback amplifier can be expressed by the standard feedback
equation
\[
A = \frac{V_{OUT}}{V_{IN}} = \frac{1 + \frac{R_2}{R_1}}{\frac{R_2}{R_1}}
\]

\subsubsection*{c)}
Take the limit as \(A_V\) goes to $\infty$ for the gain derived in part b) and compare with
that derived in EE 230 for the gain of the noninverting feedback amplifier.

\subsubsection*{d)}
Derive the gain of the inverting feedback amplifier in terms of \(A_V\) and \(\beta\) and
comment on why it does not look like the standard feedback equation.




http://class.ece.iastate.edu/ee435/miscHandouts/TSMC%200.18%20T48K.pdf

http://class.ece.iastate.edu/ee330/lectures/EE%20330%20Lect%2025%20Fall%202023.pdf

\end{document}







